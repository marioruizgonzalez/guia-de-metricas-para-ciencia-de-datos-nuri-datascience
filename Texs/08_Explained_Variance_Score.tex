\documentclass[12pt]{article}
\usepackage[utf8]{inputenc}
\usepackage{amsmath}
\usepackage{amssymb}
\usepackage{geometry}
\geometry{margin=1in}

\title{Guía de Métricas para Ciencia de Datos}
\author{Mario Ruiz}
\date{Noviembre 2025}

\begin{document}

% \maketitle

\section*{Explained Variance Score}

\subsection*{Fórmula principal}
\[
\text{Explained Variance} = 1 - \frac{\text{Var}(y - \hat{y})}{\text{Var}(y)}
\]

\noindent
Donde:
\begin{itemize}
    \item $y$: valores reales
    \item $\hat{y}$: valores predichos por el modelo
    \item $\text{Var}()$: operador de varianza
\end{itemize}

\subsection*{Desglose de componentes}
\begin{itemize}
    \item $\text{Var}(y)$: varianza total de los valores reales, representa la variabilidad inherente de los datos.
    \item $\text{Var}(y - \hat{y})$: varianza de los residuos, mide la parte de la variabilidad que el modelo no logra explicar.
    \item $\dfrac{\text{Var}(y - \hat{y})}{\text{Var}(y)}$: proporción de varianza no explicada.
    \item $1 - \dfrac{\text{Var}(y - \hat{y})}{\text{Var}(y)}$: proporción de varianza explicada por el modelo.
\end{itemize}

\subsection*{Interpretación}
El \textbf{Explained Variance Score} mide cuánta variabilidad de los datos de salida es capturada por el modelo.  
A diferencia del $R^2$, no compara el modelo con la media, sino que evalúa directamente la reducción de la varianza residual.

\subsection*{Escala de interpretación}
\begin{itemize}
    \item $1.0$: modelo perfecto (toda la varianza explicada)
    \item $0.8$–$0.9$: modelo excelente
    \item $0.6$–$0.8$: modelo bueno
    \item $0.3$–$0.6$: modelo aceptable o moderado
    \item $<0.3$: modelo pobre
    \item $<0$: modelo peor que predecir la media
\end{itemize}

\subsection*{Fundamento matemático}
El valor de esta métrica proviene de la relación entre la varianza total y la varianza de los errores.  
Su diseño permite:
\begin{itemize}
    \item Ser \textbf{invariante a la escala} de los datos.
    \item Comparar modelos en \textbf{diferentes conjuntos de datos o dominios}.
    \item Proporcionar una \textbf{medida directa de eficiencia} en la explicación de la variabilidad.
\end{itemize}

\subsection*{Resumen}
El \textit{Explained Variance Score} cuantifica la fracción de la variabilidad del objetivo que el modelo logra capturar.  
Mientras que el $R^2$ evalúa la proporción de mejora respecto a un modelo base, esta métrica se enfoca en la estabilidad y consistencia de las predicciones, siendo especialmente útil en modelos de regresión donde interesa la dispersión de los errores más que su sesgo promedio.

\end{document}
