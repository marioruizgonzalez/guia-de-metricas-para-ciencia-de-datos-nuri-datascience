\documentclass[12pt]{article}
\usepackage[utf8]{inputenc}
\usepackage{amsmath}
\usepackage{amssymb}
\usepackage{geometry}
\geometry{margin=1in}

\title{Guía de Métricas para Ciencia de Datos}
\author{Mario Ruiz}
\date{Noviembre 2025}

\begin{document}

% \maketitle  % Desactivado para no mostrar título

\section*{Root Mean Squared Error (RMSE)}

\subsection*{Fórmula del RMSE}
\[
RMSE = \sqrt{\frac{1}{n} \sum_{i=1}^{n} (y_i - \hat{y}_i)^2} = \sqrt{MSE}
\]

\noindent
Donde:
\begin{itemize}
    \item $n$: número total de observaciones
    \item $y_i$: valor real
    \item $\hat{y}_i$: valor predicho por el modelo
\end{itemize}

\subsection*{Interpretación}
El \textbf{Root Mean Squared Error (RMSE)} mide la desviación cuadrática media entre las predicciones y los valores reales.  
Su principal ventaja sobre el MSE es que devuelve el error promedio en las mismas unidades que la variable objetivo.

\subsection*{Propiedades clave}
\begin{itemize}
    \item Es siempre \textbf{no negativo}: $RMSE \ge 0$, siendo 0 el valor ideal.
    \item Usa la \textbf{distancia L2 (Euclidiana)} como referencia geométrica.
    \item Es \textbf{sensible a outliers}, pero menos que el MSE al aplicar la raíz cuadrada.
    \item Escala linealmente: $RMSE(a \times y, a \times \hat{y}) = |a| \times RMSE(y, \hat{y})$.
\end{itemize}

\subsection*{Relaciones importantes}
\[
MAE \le RMSE \le \sqrt{n} \times MAE
\]
\noindent
y su conexión con otras métricas:
\begin{itemize}
    \item $RMSE = \sqrt{MSE}$, devolviendo las unidades originales.
    \item $R^2 = 1 - \dfrac{RMSE_{modelo}^2}{RMSE_{baseline}^2}$.
\end{itemize}

\subsection*{Interpretación estadística}
Cuando los errores se distribuyen normalmente $N(0, \sigma)$:
\[
RMSE \approx \sigma
\]
\noindent
lo que significa que:
\begin{itemize}
    \item $\sim68\%$ de las predicciones están dentro de $\pm RMSE$
    \item $\sim95\%$ dentro de $\pm 2 \times RMSE$
\end{itemize}

\subsection*{Resumen}
El RMSE es una métrica universal para evaluar la precisión de modelos de regresión.  
Combina la interpretabilidad del MAE con la sensibilidad del MSE, siendo un indicador robusto del rendimiento promedio de las predicciones.

\end{document}
