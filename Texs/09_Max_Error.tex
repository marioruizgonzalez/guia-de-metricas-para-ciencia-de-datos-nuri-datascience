\documentclass[12pt]{article}
\usepackage[utf8]{inputenc}
\usepackage{amsmath}
\usepackage{amssymb}
\usepackage{geometry}
\geometry{margin=1in}

\title{Guía de Métricas para Ciencia de Datos}
\author{Mario Ruiz}
\date{Noviembre 2025}

\begin{document}

% \maketitle

\section*{Max Error}

\subsection*{Fórmula principal}
\[
\text{Max Error} = \max_{i=1,\dots,n} \, |y_i - \hat{y}_i|
\]

\noindent
Donde:
\begin{itemize}
    \item $y_i$: valor real de la observación $i$
    \item $\hat{y}_i$: valor predicho por el modelo
    \item $|\,\cdot\,|$: valor absoluto (mide la magnitud del error)
    \item $\max()$: operador que selecciona el error absoluto más grande
    \item $n$: número total de observaciones
\end{itemize}

\subsection*{Desglose matemático}
\begin{enumerate}
    \item \textbf{Error individual:} 
    \[
    e_i = |y_i - \hat{y}_i|
    \]
    Cada error representa la distancia absoluta entre el valor real y la predicción.
    
    \item \textbf{Selección del máximo:}
    \[
    \text{Max Error} = \max \{ e_1, e_2, \dots, e_n \}
    \]
    De todos los errores individuales, se elige el más grande, representando el peor desempeño del modelo.
\end{enumerate}

\subsection*{Propiedades matemáticas}
\begin{itemize}
    \item \textbf{Sensibilidad extrema a outliers:} 
    El valor está completamente determinado por una sola observación extrema.  
    Incluso si el resto de las predicciones son perfectas, un solo error grande domina el resultado.
    
    \item \textbf{No aditividad:}  
    No puede promediarse ni descomponerse como otras métricas (MAE, MSE).  
    \[
    \text{Max Error (total)} \neq \text{promedio de Max Error (subconjuntos)}
    \]
    
    \item \textbf{Monotonía:}  
    Si todos los errores disminuyen, el Max Error nunca aumenta.  
    Si un solo error empeora y se convierte en el máximo, la métrica empeora inmediatamente.
    
    \item \textbf{Invarianza al escalado:}  
    Para una constante $c > 0$,
    \[
    \text{Max Error}(c \cdot y, c \cdot \hat{y}) = c \cdot \text{Max Error}(y, \hat{y})
    \]
    manteniendo la proporcionalidad respecto a la escala de la variable objetivo.
\end{itemize}

\subsection*{Interpretación práctica}
El \textbf{Max Error} mide el peor error absoluto cometido por el modelo.  
Ofrece una visión conservadora del rendimiento, mostrando hasta qué punto el modelo puede fallar en el peor caso.

\subsection*{Por qué funciona como métrica}
\begin{itemize}
    \item Identifica de forma directa el \textit{caso extremo} del error.  
    \item Proporciona una \textbf{cota superior} del desempeño del modelo.  
    \item Se expresa en las mismas unidades que la variable objetivo, lo que facilita su interpretación.  
    \item Complementa métricas promedio (MAE, RMSE), mostrando aspectos que ellas pueden ocultar.
\end{itemize}

\subsection*{Limitaciones}
\begin{itemize}
    \item \textbf{Inestabilidad:} un solo dato atípico puede alterar completamente el valor.  
    \item \textbf{Pérdida de información:} ignora la distribución completa de errores.  
    \item \textbf{Dependencia del tamaño muestral:} tiende a aumentar con conjuntos de datos más grandes.
\end{itemize}

\subsection*{Resumen}
El \textit{Max Error} cuantifica el peor desempeño individual del modelo.  
Es una métrica útil para aplicaciones críticas o de riesgo, donde el error máximo es más relevante que el promedio, aunque su sensibilidad a valores extremos limita su uso como métrica única de evaluación.

\end{document}
