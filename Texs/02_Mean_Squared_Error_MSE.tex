\documentclass[12pt]{article}
\usepackage[utf8]{inputenc}
\usepackage{amsmath}
\usepackage{amssymb}
\usepackage{geometry}
\geometry{margin=1in}

\title{Guía de Métricas para Ciencia de Datos}
\author{Mario Ruiz}
\date{Noviembre 2025}

\begin{document}

\section*{Mean Squared Error (MSE)}

\subsection*{Fórmula del MSE}
\[
MSE = \frac{1}{n} \sum_{i=1}^{n} (y_i - \hat{y}_i)^2
\]

\noindent
Donde:
\begin{itemize}
    \item $n$: número total de observaciones
    \item $y_i$: valor real
    \item $\hat{y}_i$: valor predicho por el modelo
\end{itemize}

\subsection*{Interpretación}
El \textbf{Mean Squared Error (MSE)} mide el error cuadrático promedio entre las predicciones y los valores reales.
Cada diferencia se eleva al cuadrado, eliminando los signos y amplificando los errores grandes, lo que genera una medida global del desempeño del modelo.

\subsection*{Propiedades clave}
\begin{itemize}
    \item Es siempre \textbf{no negativa}: $MSE \ge 0$. Un valor de 0 indica predicciones perfectas.
    \item Usa la \textbf{distancia L2 (Euclidiana)} como referencia geométrica.
    \item Es \textbf{sensible a valores atípicos}, ya que los errores grandes se amplifican cuadráticamente.
    \item Es una función \textbf{convexa y diferenciable}, lo que permite optimización eficiente.
\end{itemize}

\subsection*{Relaciones importantes}
\[
MSE = \text{Bias}^2 + \text{Varianza} + \text{Error irreducible}
\]
\noindent
y su relación con otras métricas:
\begin{itemize}
    \item $RMSE = \sqrt{MSE}$ devuelve las unidades originales.
    \item $R^2$ compara el MSE del modelo con el MSE de la media.
\end{itemize}

\subsection*{Resumen}
La MSE ofrece una medida cuadrática del error promedio, útil cuando se desea penalizar con fuerza los errores grandes.
Su naturaleza convexa la convierte en una métrica ideal para algoritmos de optimización como regresión lineal y redes neuronales.

\end{document}
