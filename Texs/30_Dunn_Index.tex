\documentclass[12pt]{article}
\usepackage[utf8]{inputenc}
\usepackage{amsmath}
\usepackage{amssymb}
\usepackage{geometry}
\geometry{margin=1in}

\title{Guía de Métricas para Ciencia de Datos}
\author{Mario Ruiz}
\date{Noviembre 2025}

\begin{document}

% \maketitle

\section*{Dunn Index (DI)}

\subsection*{Definición general}
El \textbf{Dunn Index} mide la calidad de un agrupamiento considerando la relación entre la mínima distancia entre clusters y el máximo diámetro dentro de los clusters.  
Un valor alto indica una buena separación entre grupos y alta cohesión interna.

\subsection*{Fórmula general}
\[
DI = \frac{\min_{1 \le i < j \le k} \delta(C_i, C_j)}{\max_{1 \le l \le k} \Delta(C_l)}
\]

\noindent
Donde:
\begin{itemize}
    \item $k$: Número total de clusters.
    \item $C_i, C_j$: Clusters $i$ y $j$ (con $i \neq j$).
    \item $\delta(C_i, C_j)$: Distancia mínima entre clusters $i$ y $j$ (separación inter-cluster).
    \item $\Delta(C_l)$: Diámetro máximo del cluster $l$ (compacidad intra-cluster).
\end{itemize}

\subsection*{Componentes matemáticos}

\paragraph{1. Separación inter-cluster}
\[
\delta(C_i, C_j) = \min \{ d(x, y) : x \in C_i, \, y \in C_j \}
\]
\noindent
\textit{Interpretación:} Es la menor distancia entre cualquier par de puntos pertenecientes a clusters distintos.  
Mide qué tan bien separados están los clusters.

\paragraph{2. Compacidad intra-cluster}
\[
\Delta(C_l) = \max \{ d(x, y) : x, y \in C_l \}
\]
\noindent
\textit{Interpretación:} Es la máxima distancia entre dos puntos dentro del mismo cluster.  
Mide la dispersión interna o “diámetro” del cluster.

\subsection*{Interpretación matemática}
\begin{itemize}
    \item \textbf{Numerador:} Representa la mínima separación entre todos los pares de clusters ($\min \delta(C_i, C_j)$).
    \item \textbf{Denominador:} Representa la máxima dispersión interna entre los clusters ($\max \Delta(C_l)$).
    \item \textbf{Resultado:} Cuanto mayor sea el cociente, mejor es la estructura del clustering.
\end{itemize}

\subsection*{Propiedades y rango}
\begin{itemize}
    \item \textbf{Rango:} $DI \in [0, +\infty)$.
    \item \textbf{Interpretación:}  
    Valores grandes indican clusters bien separados y compactos.
    \item \textbf{Valor ideal:} No tiene un máximo teórico; se busca maximizar el índice.
\end{itemize}

\subsection*{Fundamento teórico}
El \textbf{Dunn Index} combina dos objetivos opuestos:
\begin{enumerate}
    \item Maximizar la \textbf{separación entre clusters} ($\delta$).
    \item Minimizar la \textbf{dispersión interna} ($\Delta$).
\end{enumerate}
Esto garantiza que los grupos formados sean simultáneamente compactos y bien definidos.

\subsection*{Propiedades matemáticas clave}
\begin{itemize}
    \item \textbf{Normalización implícita:} La relación $\delta / \Delta$ lo hace independiente de la escala de los datos.
    \item \textbf{Sensibilidad:} Penaliza fuerte a agrupamientos con solapamientos o clusters dispersos.
    \item \textbf{Invarianza a traslaciones:} Desplazar los datos no cambia el índice.
\end{itemize}

\subsection*{Ventajas e interpretación práctica}
\begin{itemize}
    \item Ideal para evaluar la separación relativa entre clusters.
    \item Útil como criterio de selección del número óptimo de clusters $k$.
    \item Un Dunn Index bajo sugiere solapamiento o mala compactación de grupos.
\end{itemize}

\subsection*{Conclusión}
El \textbf{Dunn Index} ofrece una métrica intuitiva para evaluar la calidad de un clustering mediante una simple relación entre cohesión y separación.  
Su formulación geométrica y su independencia de escala lo convierten en un complemento ideal al \textit{Silhouette}, \textit{Davies–Bouldin} y \textit{Calinski–Harabasz} para una evaluación integral de modelos no supervisados.

\end{document}
