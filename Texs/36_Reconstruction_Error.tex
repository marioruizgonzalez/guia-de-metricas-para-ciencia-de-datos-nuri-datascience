\documentclass[12pt]{article}
\usepackage[utf8]{inputenc}
\usepackage{amsmath}
\usepackage{amssymb}
\usepackage{geometry}
\geometry{margin=1in}

\title{Guía de Métricas para Ciencia de Datos}
\author{Mario Ruiz}
\date{Noviembre 2025}

\begin{document}

% \maketitle

\section*{Reconstruction Error}

\subsection*{Definición general}
El \textbf{Reconstruction Error (RE)} cuantifica la pérdida de información entre los datos originales y su reconstrucción.  
Mide la distancia entre la matriz original $X$ y su versión reconstruida $\hat{X}$, utilizando la norma euclidiana al cuadrado.

\subsection*{Fórmula principal}
\[
RE = \| X - \hat{X} \|^2
\]

\noindent
Donde:
\begin{itemize}
    \item $X$: matriz de datos originales $(n \times d)$.
    \item $\hat{X}$: matriz reconstruida $(n \times d)$.
    \item $\| \cdot \|^2$: norma euclidiana al cuadrado.
\end{itemize}

\subsection*{Desglose por componentes}
\paragraph{1. Error por muestra}
\[
RE_i = \sum_{j=1}^{d} (x_{ij} - \hat{x}_{ij})^2
\]

\paragraph{2. Error promedio}
\[
RE_{\text{mean}} = \frac{1}{n} \sum_{i=1}^{n} RE_i
\]

\paragraph{3. Error relativo (normalizado)}
\[
RE_{\text{rel}} = \frac{\| X - \hat{X} \|^2}{\| X \|^2}
\]

\subsection*{Aplicación en PCA}
Para PCA con $k$ componentes principales:
\[
Z = X W_k
\]
\[
\hat{X} = Z W_k^T = X W_k W_k^T
\]
\[
RE = \| X - X W_k W_k^T \|^2
\]

\noindent
Donde:
\begin{itemize}
    \item $W_k$: matriz de los primeros $k$ eigenvectores $(d \times k)$.
    \item $Z$: proyección de $X$ al subespacio reducido $(n \times k)$.
\end{itemize}

\subsection*{Interpretación matemática}
\begin{itemize}
    \item \textbf{Interpretación geométrica:} El $RE$ mide la distancia entre un punto original y su proyección reconstruida en el subespacio de menor dimensión.
    \item \textbf{Relación con la varianza:} En PCA, el error de reconstrucción está ligado a la \textit{varianza no explicada}:
    \[
    \text{Varianza no explicada} = \sum_{i=k+1}^{d} \lambda_i
    \]
    \item \textbf{Propiedad de optimalidad:} PCA minimiza el $RE$ entre todas las proyecciones lineales de rango $k$, siendo la solución óptima de mínima pérdida.
\end{itemize}

\subsection*{Por qué funciona como métrica}
\begin{itemize}
    \item \textbf{Sensibilidad a pérdida de información:} Refleja directamente cuánta información se descarta en la reducción dimensional.
    \item \textbf{Comparabilidad:} Permite comparar distintas técnicas de reducción (PCA, Autoencoders, SVD, etc.).
    \item \textbf{Selección de parámetros:} Ayuda a determinar el número óptimo de componentes conservando un nivel deseado de reconstrucción.
    \item \textbf{Detección de anomalías:} Errores elevados pueden señalar observaciones atípicas o estructuras no capturadas por el modelo.
\end{itemize}

\subsection*{Conclusión}
El \textbf{Reconstruction Error} es una métrica fundamental para evaluar la eficiencia de modelos de reducción dimensional.  
Su naturaleza geométrica y relación directa con la varianza perdida la convierten en una herramienta esencial para analizar la calidad de representaciones comprimidas, tanto en PCA como en autoencoders y otros métodos de aprendizaje no supervisado.

\end{document}
