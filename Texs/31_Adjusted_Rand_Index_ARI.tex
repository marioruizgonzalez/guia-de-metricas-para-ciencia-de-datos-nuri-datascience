\documentclass[12pt]{article}
\usepackage[utf8]{inputenc}
\usepackage{amsmath}
\usepackage{amssymb}
\usepackage{geometry}
\geometry{margin=1in}

\title{Guía de Métricas para Ciencia de Datos}
\author{Mario Ruiz}
\date{Noviembre 2025}

\begin{document}

% \maketitle

\section*{Adjusted Rand Index (ARI)}

\subsection*{Definición general}
El \textbf{Adjusted Rand Index (ARI)} mide la similitud entre dos particiones de un conjunto de datos (por ejemplo, entre etiquetas verdaderas y clusters obtenidos).  
A diferencia del Rand Index clásico, el ARI \textit{corrige por el azar}, proporcionando una evaluación más justa de la concordancia entre agrupamientos.

\subsection*{Fórmula general}
\[
ARI = \frac{RI - \mathbb{E}[RI]}{\max(RI) - \mathbb{E}[RI]}
\]
\noindent
Donde:
\begin{itemize}
    \item $RI$: Rand Index observado.
    \item $\mathbb{E}[RI]$: Valor esperado del Rand Index bajo hipótesis nula (agrupamiento aleatorio).
    \item $\max(RI)$: Valor máximo posible del Rand Index.
\end{itemize}

\subsection*{Construcción de la tabla de contingencia}
Sea una tabla de contingencia que cruza los clusters reales ($R_i$) con los predichos ($C_j$):

\[
\begin{array}{c|cccc|c}
 & C_1 & C_2 & \cdots & C_j & \text{Totales} \\
\hline
R_1 & n_{11} & n_{12} & \cdots & n_{1j} & a_1 \\
R_2 & n_{21} & n_{22} & \cdots & n_{2j} & a_2 \\
\vdots & \vdots & \vdots & \ddots & \vdots & \vdots \\
R_i & n_{i1} & n_{i2} & \cdots & n_{ij} & a_i \\
\hline
\text{Totales} & b_1 & b_2 & \cdots & b_j & n \\
\end{array}
\]

\noindent
Donde:
\begin{itemize}
    \item $n_{ij}$: Número de puntos en la intersección entre el cluster real $R_i$ y el predicho $C_j$.
    \item $a_i = \sum_j n_{ij}$: Total de puntos en el cluster real $R_i$.
    \item $b_j = \sum_i n_{ij}$: Total de puntos en el cluster predicho $C_j$.
\end{itemize}

\subsection*{Cálculo del Rand Index}
El \textbf{Rand Index (RI)} mide la proporción de pares de puntos que son clasificados consistentemente en ambas particiones:
\[
RI = \frac{a + b}{\binom{n}{2}}
\]
\noindent
Donde:
\begin{itemize}
    \item $a$: Número de pares en el mismo cluster en ambas particiones.
    \item $b$: Número de pares en clusters distintos en ambas particiones.
    \item $\binom{n}{2} = \frac{n(n-1)}{2}$: Total de pares posibles.
\end{itemize}

\subsection*{Fórmula expandida del ARI}
\[
ARI = 
\frac{
    \sum_{i,j} \binom{n_{ij}}{2}
    - \frac{
        \sum_i \binom{a_i}{2} \sum_j \binom{b_j}{2}
      }{\binom{n}{2}}
}{
    \frac{1}{2}
    \left[
        \sum_i \binom{a_i}{2} + \sum_j \binom{b_j}{2}
    \right]
    - \frac{
        \sum_i \binom{a_i}{2} \sum_j \binom{b_j}{2}
      }{\binom{n}{2}}
}
\]

\noindent
\textbf{Significado de los términos:}
\begin{itemize}
    \item $\binom{n_{ij}}{2}$: Pares de puntos que coinciden en $R_i$ y $C_j$.
    \item $\binom{a_i}{2}$: Pares en el cluster $R_i$ del conjunto real.
    \item $\binom{b_j}{2}$: Pares en el cluster $C_j$ del conjunto predicho.
\end{itemize}

\noindent
\textbf{Numerador:} Acuerdos observados menos acuerdos esperados por azar.  
\textbf{Denominador:} Máximos acuerdos posibles menos acuerdos esperados por azar.

\subsection*{Propiedades matemáticas}
\begin{itemize}
    \item \textbf{Rango:} $ARI \in [-1, 1]$
    \item \textbf{Interpretación:}
    \begin{itemize}
        \item $ARI = 1$: Concordancia perfecta.
        \item $ARI = 0$: Concordancia equivalente al azar.
        \item $ARI < 0$: Peor que el azar (raro en la práctica).
    \end{itemize}
    \item \textbf{Simetría:} $ARI(X, Y) = ARI(Y, X)$.
    \item \textbf{Corrección por azar:} Elimina el sesgo hacia valores altos cuando hay muchos clusters pequeños.
\end{itemize}

\subsection*{Ventajas e interpretación práctica}
\begin{itemize}
    \item Robusto ante el número de clusters.
    \item Adecuado para comparar resultados de diferentes algoritmos de clustering con etiquetas conocidas.
    \item Más informativo que el Rand Index simple.
\end{itemize}

\subsection*{Conclusión}
El \textbf{Adjusted Rand Index} proporciona una medida estadísticamente sólida de similitud entre particiones.  
Su ajuste por azar y su rango interpretativo lo hacen una herramienta esencial para evaluar modelos de clustering supervisado y semisupervisado.

\end{document}
