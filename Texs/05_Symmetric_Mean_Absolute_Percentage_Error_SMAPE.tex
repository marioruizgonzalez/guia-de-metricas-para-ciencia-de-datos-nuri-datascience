\documentclass[12pt]{article}
\usepackage[utf8]{inputenc}
\usepackage{amsmath}
\usepackage{amssymb}
\usepackage{geometry}
\geometry{margin=1in}

\title{Guía de Métricas para Ciencia de Datos}
\author{Mario Ruiz}
\date{Noviembre 2025}

\begin{document}

% \maketitle

\section*{Symmetric Mean Absolute Percentage Error (SMAPE)}

\subsection*{Fórmula del SMAPE}
\[
SMAPE = \frac{2}{n} \sum_{i=1}^{n} 
\frac{|y_i - \hat{y}_i|}{|y_i| + |\hat{y}_i|} \times 100
\]

\noindent
Donde:
\begin{itemize}
    \item $n$: número total de observaciones
    \item $y_i$: valor real
    \item $\hat{y}_i$: valor predicho
    \item $|y_i - \hat{y}_i|$: error absoluto
    \item $|y_i| + |\hat{y}_i|$: suma de magnitudes reales y predichas (crea la simetría)
\end{itemize}

\subsection*{Interpretación}
El \textbf{Symmetric Mean Absolute Percentage Error (SMAPE)} mide el error medio relativo expresado en porcentaje, corrigiendo el sesgo asimétrico del MAPE mediante una normalización simétrica en el denominador.

\subsection*{Simetría matemática}
Para valores positivos $y$ y $\hat{y}$:
\[
SMAPE = \frac{2|y - \hat{y}|}{|y| + |\hat{y}|} \times 100
\]
\noindent
Si $\hat{y} = k y$ (sobreestimación) o $\hat{y} = y/k$ (subestimación) con $k > 1$, se cumple:
\[
SMAPE = \frac{2(k-1)}{1+k} \times 100
\]
\noindent
demostrando que el error es simétrico para sobre y subestimaciones del mismo factor.

\subsection*{Propiedades clave}
\begin{itemize}
    \item \textbf{Rango:} $[0, 200\%]$
    \item \textbf{Simetría:} Trata por igual sobreestimaciones y subestimaciones.
    \item \textbf{Unidades:} Porcentaje (\%).
    \item \textbf{Estabilidad:} No diverge hacia infinito como el MAPE cuando $y_i \to 0$.
\end{itemize}

\subsection*{Comportamiento límite}
\begin{itemize}
    \item Si $y_i = 0$ y $\hat{y}_i > 0$, entonces $SMAPE = 200\%$.
    \item Si $y_i > 0$ y $\hat{y}_i = 0$, también $SMAPE = 200\%$.
    \item Cuando $y_i \to 0$ y $\hat{y}_i \to 0$, la expresión se vuelve indeterminada ($0/0$).
\end{itemize}

\subsection*{Ventajas frente a MAPE}
\begin{itemize}
    \item Mantiene el error \textbf{acotado} entre 0 y 200\%.
    \item \textbf{Simétrico} en su tratamiento de errores relativos.
    \item \textbf{Robusto} ante valores negativos o series con magnitudes distintas.
\end{itemize}

\subsection*{Resumen}
El SMAPE ofrece una alternativa más equilibrada al MAPE, especialmente en contextos donde existen valores pequeños o negativos.  
Su estructura simétrica evita sesgos de escala y proporciona una interpretación porcentual consistente del error relativo medio.

\end{document}
