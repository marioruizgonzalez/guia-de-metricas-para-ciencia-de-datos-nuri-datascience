\documentclass[12pt]{article}
\usepackage[utf8]{inputenc}
\usepackage{amsmath}
\usepackage{amssymb}
\usepackage{geometry}
\geometry{margin=1in}

\title{Guía de Métricas para Ciencia de Datos}
\author{Mario Ruiz}
\date{Noviembre 2025}

\begin{document}

% \maketitle

\section*{V-measure}

\subsection*{Definición general}
La \textbf{V-measure} evalúa la calidad de un clustering combinando dos propiedades fundamentales:  
la \textit{homogeneidad} (pureza de los clusters) y la \textit{completitud} (agrupamiento total de cada clase).  
Se define como la \textbf{media armónica} entre ambas métricas, análoga al \textbf{F1-score} en clasificación supervisada.

\subsection*{Fórmula principal}
\[
V = \frac{2 \times \text{Homogeneity} \times \text{Completeness}}
{\text{Homogeneity} + \text{Completeness}}
\]

\noindent
Donde:
\begin{itemize}
    \item \textbf{Homogeneity:} mide si cada cluster contiene solo miembros de una clase.
    \item \textbf{Completeness:} mide si todos los miembros de una clase están asignados al mismo cluster.
\end{itemize}

\subsection*{Componentes matemáticos}

\paragraph{1. Homogeneidad}
\[
\text{Homogeneity} = 1 - \frac{H(C|K)}{H(C)}
\]
\noindent
\textit{Significado:} mide la pureza de los clusters.  
Si cada cluster contiene elementos de una sola clase, $H(C|K) = 0$ y por tanto la homogeneidad es máxima.

\paragraph{2. Completitud}
\[
\text{Completeness} = 1 - \frac{H(K|C)}{H(K)}
\]
\noindent
\textit{Significado:} mide el grado en que cada clase está contenida dentro de un único cluster.

\subsection*{Entropías utilizadas}
\[
H(C) = -\sum_{i=1}^{|C|} \frac{|C_i|}{N} \log \left( \frac{|C_i|}{N} \right)
\]
\[
H(C|K) = -\sum_{k=1}^{|K|} \sum_{c=1}^{|C|} 
\frac{|C_k \cap C_c|}{N} 
\log \left( \frac{|C_k \cap C_c|}{|K_k|} \right)
\]
\noindent
Donde:
\begin{itemize}
    \item $C_i$: subconjunto de la clase $i$.
    \item $K_k$: subconjunto del cluster $k$.
    \item $|C_k \cap C_c|$: número de muestras compartidas entre clase $c$ y cluster $k$.
    \item $N$: número total de muestras.
\end{itemize}

\subsection*{Interpretación matemática}
\begin{itemize}
    \item $V = 1$: Clustering perfecto (homogeneidad y completitud iguales a 1).
    \item $V = 0$: Clustering completamente aleatorio (sin estructura informativa).
    \item $0 < V < 1$: Grado intermedio de calidad del clustering.
\end{itemize}

\subsection*{Propiedades teóricas}
\begin{itemize}
    \item \textbf{Media armónica:} penaliza fuertemente los desequilibrios; si una de las dos métricas es baja, $V$ también lo será.
    \item \textbf{Basada en entropía:} se fundamenta en la teoría de la información, utilizando incertidumbre como medida de error.
    \item \textbf{Normalización:} siempre acotada en $[0,1]$, lo que facilita la comparación entre distintos modelos o datasets.
    \item \textbf{Simetría:} $V(C,K) = V(K,C)$, propiedad deseable para comparar particiones.
\end{itemize}

\subsection*{Conclusión}
La \textbf{V-measure} resume la calidad global de un clustering al equilibrar la pureza (homogeneidad) y la cobertura (completitud).  
Su estructura armónica garantiza una evaluación justa, penalizando tanto la mezcla de clases dentro de clusters como la fragmentación de las clases verdaderas entre múltiples clusters.

\end{document}
