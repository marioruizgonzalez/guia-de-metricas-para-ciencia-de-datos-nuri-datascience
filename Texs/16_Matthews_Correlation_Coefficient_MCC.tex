\documentclass[12pt]{article}
\usepackage[utf8]{inputenc}
\usepackage{amsmath}
\usepackage{amssymb}
\usepackage{geometry}
\geometry{margin=1in}

\title{Guía de Métricas para Ciencia de Datos}
\author{Mario Ruiz}
\date{Noviembre 2025}

\begin{document}

% \maketitle

\section*{Matthews Correlation Coefficient (MCC)}

\subsection*{Fórmula principal}
\[
\text{MCC} = \frac{TP \times TN - FP \times FN}{\sqrt{(TP + FP)(TP + FN)(TN + FP)(TN + FN)}}
\]

\subsection*{Componentes de la fórmula}
\begin{itemize}
    \item $TP$: Verdaderos positivos — casos correctamente clasificados como positivos.
    \item $TN$: Verdaderos negativos — casos correctamente clasificados como negativos.
    \item $FP$: Falsos positivos — negativos clasificados como positivos.
    \item $FN$: Falsos negativos — positivos clasificados como negativos.
\end{itemize}

\subsection*{Interpretación matemática}
\textbf{Numerador:}
\[
TP \times TN - FP \times FN
\]
Representa la diferencia entre las predicciones correctas $(TP \times TN)$ y los errores cruzados $(FP \times FN)$.
\begin{itemize}
    \item Si $FP = FN = 0$, entonces el numerador = $TP \times TN$ (predicción perfecta).
    \item Si $TP = TN = 0$, entonces el numerador = $-FP \times FN$ (predicción completamente incorrecta).
\end{itemize}

\textbf{Denominador:}
\[
\sqrt{(TP + FP)(TP + FN)(TN + FP)(TN + FN)}
\]
Corresponde al producto geométrico de los marginales de la matriz de confusión.  
Actúa como factor de normalización que asegura que el MCC esté limitado en el rango:
\[
-1 \leq \text{MCC} \leq +1
\]

\subsection*{Propiedades matemáticas y fundamentos}
\begin{itemize}
    \item \textbf{Simetría:} Trata ambas clases por igual, sin sesgo hacia la clase mayoritaria.
    \item \textbf{Normalización:} Ajusta el valor en función de la distribución de clases, permitiendo comparaciones entre datasets.
    \item \textbf{Fundamento estadístico:} Equivale al coeficiente de correlación de Pearson entre las variables binarias \textit{real} y \textit{predicha}.
    \item \textbf{Robustez:} Su interpretación se mantiene consistente incluso en contextos con fuerte desbalance de clases.
\end{itemize}

\subsection*{Casos especiales}
\begin{itemize}
    \item Si alguno de los marginales $(TP+FP)$, $(TP+FN)$, $(TN+FP)$ o $(TN+FN)$ es cero, entonces $\text{MCC} = 0$.
    \item $\text{MCC} = +1$ cuando $FP = FN = 0$ (predicción perfecta).
    \item $\text{MCC} = -1$ cuando $TP = TN = 0$ (predicción perfectamente incorrecta).
\end{itemize}

\subsection*{Conclusión}
El \textbf{Matthews Correlation Coefficient} es una métrica estadísticamente sólida para evaluar modelos de clasificación binaria.  
A diferencia del \textit{Accuracy}, incorpora toda la información de la matriz de confusión y proporciona una visión equilibrada del rendimiento, incluso en datasets desbalanceados.  
Por ello, se considera una de las métricas más completas para evaluar el poder discriminante de un modelo.

\end{document}
