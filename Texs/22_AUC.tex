\documentclass[12pt]{article}
\usepackage[utf8]{inputenc}
\usepackage{amsmath}
\usepackage{amssymb}
\usepackage{geometry}
\geometry{margin=1in}

\title{Guía de Métricas para Ciencia de Datos}
\author{Mario Ruiz}
\date{Noviembre 2025}

\begin{document}

% \maketitle

\section*{AUC (Área Bajo la Curva ROC)}

\subsection*{Definición general}
La \textbf{AUC (Area Under the ROC Curve)} mide la capacidad de un modelo de clasificación para distinguir entre clases positivas y negativas a través de todos los posibles umbrales de decisión.

\subsection*{Construcción de la Curva ROC}
La \textbf{Curva ROC} se grafica a partir de dos tasas fundamentales:

\[
\text{Eje Y: } TPR = \frac{TP}{TP + FN} \quad \text{(Sensibilidad / Recall)}
\]
\[
\text{Eje X: } FPR = \frac{FP}{FP + TN} = 1 - \text{Especificidad}
\]

\noindent
Para cada umbral $t$ de probabilidad:
\[
\hat{y} =
\begin{cases}
1, & \text{si } P(y=1|x) \geq t \\
0, & \text{en caso contrario}
\end{cases}
\]

\subsection*{Formulación matemática del AUC}

\paragraph{1. Definición integral:}
\[
AUC = \int_{0}^{1} TPR(FPR^{-1}(x)) \, dx
\]

\paragraph{2. Interpretación probabilística:}
\[
AUC = P(\text{score}(x_+) > \text{score}(x_-))
\]
donde $x_+$ es un ejemplo positivo y $x_-$ un ejemplo negativo.  
\textbf{Interpretación:} El AUC representa la probabilidad de que un ejemplo positivo reciba un puntaje mayor que un ejemplo negativo elegido al azar.

\paragraph{3. Cálculo discreto (Fórmula trapezoidal):}
\[
AUC = \sum_{i=1}^{n-1} (FPR_{i+1} - FPR_i) \times \frac{(TPR_{i+1} + TPR_i)}{2}
\]
Esta aproximación numérica se usa comúnmente para calcular el área bajo la curva en implementaciones computacionales.

\subsection*{Componentes explicados}
\begin{itemize}
    \item \textbf{TPR (Recall / Sensibilidad):}  
    Proporción de positivos reales correctamente identificados por el modelo.
    \item \textbf{FPR (False Positive Rate):}  
    Proporción de negativos reales clasificados erróneamente como positivos.
    \item \textbf{Umbral ($t$):}  
    Punto de corte sobre la probabilidad estimada que define la frontera entre clases.
\end{itemize}

\subsection*{Fundamento matemático}
\begin{itemize}
    \item \textbf{Invariancia al umbral:}  
    Evalúa el rendimiento del modelo sobre todos los posibles valores de $t$ simultáneamente.
    \item \textbf{Interpretación geométrica:}  
    El área bajo la curva ROC refleja la capacidad del modelo para ordenar correctamente pares positivo-negativo.
    \item \textbf{Conexión estadística:}  
    El AUC es equivalente al estadístico de \textit{Mann–Whitney U}:
    \[
    AUC = \frac{U_1}{n_1 n_2}
    \]
    donde $n_1$ y $n_2$ son los tamaños de las clases positiva y negativa respectivamente.
\end{itemize}

\subsection*{Conclusión}
El \textbf{AUC} es una métrica robusta y ampliamente interpretada que resume en un solo valor la habilidad discriminativa del modelo, independientemente del umbral.  
Su interpretación probabilística y su relación con pruebas estadísticas no paramétricas la convierten en una medida esencial para la evaluación de modelos de clasificación binaria.

\end{document}
