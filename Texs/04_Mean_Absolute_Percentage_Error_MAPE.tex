\documentclass[12pt]{article}
\usepackage[utf8]{inputenc}
\usepackage{amsmath}
\usepackage{amssymb}
\usepackage{geometry}
\geometry{margin=1in}

\title{Guía de Métricas para Ciencia de Datos}
\author{Mario Ruiz}
\date{Noviembre 2025}

\begin{document}

% \maketitle  % Desactivado para mantener formato limpio

\section*{Mean Absolute Percentage Error (MAPE)}

\subsection*{Fórmula del MAPE}
\[
MAPE = \frac{100}{n} \sum_{i=1}^{n} \left| \frac{y_i - \hat{y}_i}{y_i} \right|
\]

\noindent
Donde:
\begin{itemize}
    \item $n$: número total de observaciones
    \item $y_i$: valor real (ground truth)
    \item $\hat{y}_i$: valor predicho por el modelo
    \item $\dfrac{y_i - \hat{y}_i}{y_i}$: error relativo (proporción respecto al valor real)
\end{itemize}

\subsection*{Interpretación}
El \textbf{Mean Absolute Percentage Error (MAPE)} mide el error medio en porcentaje entre los valores reales y los predichos.  
Expresa, en promedio, cuánto se desvía la predicción del valor verdadero, normalizando cada error por el valor real.

\subsection*{Propiedades matemáticas}
\begin{itemize}
    \item \textbf{Rango:} $[0, +\infty)$, donde 0 indica predicciones perfectas.
    \item \textbf{Unidades:} Porcentaje (\%).
    \item \textbf{Simetría:} No simétrica, ya que los errores relativos dependen del denominador $y_i$.
    \item \textbf{Sensibilidad:} Mayor impacto de errores en valores pequeños que en grandes.
\end{itemize}

\subsection*{Comportamiento y limitaciones}
\begin{itemize}
    \item Cuando $y_i \to 0$, el MAPE $\to \infty$, lo que lo hace inadecuado para series con valores cercanos a cero.
    \item Penaliza de forma proporcional: un error del 10\% tiene el mismo peso en cualquier punto de la escala.
\end{itemize}

\subsection*{Resumen}
El MAPE es una métrica intuitiva, expresada en porcentaje, que facilita la interpretación del error.  
Sin embargo, su sensibilidad a valores pequeños y su falta de simetría limitan su uso en contextos donde los datos pueden tomar valores cercanos a cero.

\end{document}
