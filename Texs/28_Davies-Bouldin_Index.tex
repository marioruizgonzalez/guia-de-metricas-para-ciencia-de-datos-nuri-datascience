\documentclass[12pt]{article}
\usepackage[utf8]{inputenc}
\usepackage{amsmath}
\usepackage{amssymb}
\usepackage{geometry}
\geometry{margin=1in}

\title{Guía de Métricas para Ciencia de Datos}
\author{Mario Ruiz}
\date{Noviembre 2025}

\begin{document}

% \maketitle

\section*{Davies–Bouldin Index (DBI)}

\subsection*{Definición general}
El \textbf{Davies–Bouldin Index (DBI)} evalúa la calidad de un agrupamiento midiendo la relación entre la \textit{dispersión interna} de los clusters y su \textit{separación mutua}.  
Un valor bajo de DBI indica clusters compactos y bien separados, lo que refleja un agrupamiento de alta calidad.

\subsection*{Fórmula general}
\[
DBI = \frac{1}{k} \sum_{i=1}^{k} \max_{j \neq i} R(i, j)
\]
\noindent
Donde $k$ es el número total de clusters y $R(i,j)$ mide la \textit{similaridad} entre los clusters $i$ y $j$:

\[
R(i,j) = \frac{S(i) + S(j)}{M(i,j)}
\]

\subsection*{Componentes matemáticos}

\paragraph{1. Dispersión intra-cluster (cohesión)}
\[
S(i) = \frac{1}{|C_i|} \sum_{x \in C_i} \| x - \mu_i \|_p
\]
\noindent
Donde:
\begin{itemize}
    \item $C_i$: Conjunto de puntos en el cluster $i$.
    \item $|C_i|$: Número de puntos en el cluster $i$.
    \item $\mu_i$: Centroide del cluster $i$.
    \item $\| \cdot \|_p$: Norma $L_p$ (usualmente $p=2$, norma euclidiana).
\end{itemize}

\noindent
\textit{Interpretación:} $S(i)$ mide qué tan "apretados" están los puntos alrededor de su centroide.  
Valores pequeños indican alta cohesión.

\paragraph{2. Distancia inter-cluster (separación)}
\[
M(i,j) = \| \mu_i - \mu_j \|_p
\]
\noindent
\textit{Interpretación:} Mide la distancia entre los centroides de los clusters $i$ y $j$.  
Valores grandes implican clusters bien separados.

\paragraph{3. Razón de similaridad entre clusters}
\[
R(i,j) = \frac{S(i) + S(j)}{M(i,j)}
\]
\noindent
\textit{Interpretación:} Relación entre la suma de dispersión interna de dos clusters y la distancia entre ellos.  
Un valor bajo indica buena separación y compactación.

\paragraph{4. Cálculo final del DBI}
Para cada cluster $i$, se identifica el cluster $j$ que maximiza $R(i,j)$ (el más similar), y luego se promedia este valor sobre todos los clusters:
\[
DBI = \frac{1}{k} \sum_{i=1}^{k} \max_{j \neq i} R(i,j)
\]

\subsection*{Interpretación matemática}
\begin{itemize}
    \item \textbf{Caso ideal ($DBI \to 0$):}  
    Clusters perfectamente compactos y bien separados.
    \item \textbf{Valores moderados ($1.0 \le DBI \le 2.0$):}  
    Clusters con cierto solapamiento pero estructura visible.
    \item \textbf{Valores altos ($DBI > 2.0$):}  
    Clusters difusos o mal definidos.
\end{itemize}

\subsection*{Fundamento teórico}
El DBI combina dos objetivos matemáticos opuestos:
\begin{enumerate}
    \item Minimizar la \textbf{dispersión interna} $S(i)$.
    \item Maximizar la \textbf{separación entre clusters} $M(i,j)$.
\end{enumerate}
De esta forma, un buen modelo de clustering tendrá valores pequeños de $R(i,j)$ para todos los pares $(i,j)$, reduciendo el promedio global del índice.

\subsection*{Propiedades matemáticas}
\begin{itemize}
    \item \textbf{Rango:} $DBI \ge 0$ (cuanto menor, mejor).
    \item \textbf{Invarianza a escala:} Escalar o normalizar los datos no altera el valor de $DBI$.
    \item \textbf{Sensibilidad:}  
    - Funciona mejor con clusters convexos y de tamaño similar.  
    - Sensible a outliers que distorsionan los centroides.
\end{itemize}

\subsection*{Interpretación práctica}
\begin{center}
\begin{tabular}{|c|l|}
\hline
\textbf{Valor DBI} & \textbf{Interpretación} \\
\hline
$< 1.0$ & Clusters compactos y bien separados \\
$1.0$–$2.0$ & Clusters con leve solapamiento \\
$> 2.0$ & Clusters mal definidos o mezclados \\
\hline
\end{tabular}
\end{center}

\subsection*{Limitaciones}
\begin{itemize}
    \item \textbf{Sesgo hacia clusters convexos:}  
    El uso de distancia euclidiana limita su desempeño en clusters con formas no convexas.
    \item \textbf{Sensibilidad a outliers:}  
    Puntos extremos pueden alterar significativamente los centroides.
    \item \textbf{Suposición de tamaños similares:}  
    Clusters muy desbalanceados pueden producir valores de DBI engañosos.
\end{itemize}

\subsection*{Conclusión}
El \textbf{Davies–Bouldin Index} es una métrica geométrica que equilibra cohesión y separación en un mismo marco analítico.  
Su simplicidad, interpretación directa y compatibilidad con diferentes métricas de distancia lo hacen ideal para evaluar y comparar resultados de clustering no supervisado.

\end{document}
