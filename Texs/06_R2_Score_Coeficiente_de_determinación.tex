\documentclass[12pt]{article}
\usepackage[utf8]{inputenc}
\usepackage{amsmath}
\usepackage{amssymb}
\usepackage{geometry}
\geometry{margin=1in}

\title{Guía de Métricas para Ciencia de Datos}
\author{Mario Ruiz}
\date{Noviembre 2025}

\begin{document}

% \maketitle

\section*{R² Score (Coeficiente de Determinación)}

\subsection*{Fórmula principal}
\[
R^2 = 1 - \frac{SS_{res}}{SS_{tot}}
\]

\noindent
Donde:
\begin{itemize}
    \item $SS_{res} = \sum_{i=1}^{n} (y_i - \hat{y}_i)^2$ \quad (Suma de cuadrados de los residuos)
    \item $SS_{tot} = \sum_{i=1}^{n} (y_i - \bar{y})^2$ \quad (Suma total de cuadrados)
\end{itemize}

\subsection*{Interpretación matemática}
El \textbf{R² Score} mide la proporción de la varianza total de los datos que el modelo logra explicar.  
Su fórmula puede reescribirse como:
\[
R^2 = 1 - \frac{\sum (y_i - \hat{y}_i)^2}{\sum (y_i - \bar{y})^2}
\]
\noindent
donde el numerador representa el error del modelo y el denominador la variabilidad total del conjunto de datos.

\subsection*{Desglose de componentes}
\begin{itemize}
    \item \textbf{$SS_{res}$:} mide el error residual, es decir, cuánto difieren las predicciones de los valores reales.
    \item \textbf{$SS_{tot}$:} mide la variabilidad total respecto a la media $\bar{y}$.
    \item La razón $\dfrac{SS_{res}}{SS_{tot}}$ representa la fracción de la varianza que el modelo no logra explicar.
\end{itemize}

\subsection*{Interpretación práctica}
\[
R^2 = 1 - \frac{SS_{res}}{SS_{tot}}
\]
\noindent
\begin{itemize}
    \item Si $SS_{res} = 0$: $R^2 = 1$ → predicciones perfectas.
    \item Si $SS_{res} = SS_{tot}$: $R^2 = 0$ → el modelo no mejora sobre predecir la media.
    \item Si $SS_{res} > SS_{tot}$: $R^2 < 0$ → el modelo es peor que usar la media como predictor.
\end{itemize}

\subsection*{Propiedades matemáticas}
\begin{itemize}
    \item \textbf{Rango:} $(-\infty, 1]$
    \item \textbf{Interpretación geométrica:} $R^2$ mide la proporción de varianza explicada por la proyección del vector de observaciones sobre el espacio generado por las predicciones.
    \item \textbf{Relación con la correlación lineal:} Para modelos lineales simples,
    \[
    R^2 = r^2
    \]
    donde $r$ es el coeficiente de correlación de Pearson.
\end{itemize}

\subsection*{Resumen}
El coeficiente de determinación $R^2$ indica qué tan bien un modelo explica la variabilidad de los datos.  
Un valor cercano a 1 significa un ajuste excelente, mientras que valores cercanos a 0 o negativos sugieren un modelo ineficiente o mal especificado.

\end{document}
