\documentclass[12pt]{article}
\usepackage[utf8]{inputenc}
\usepackage{amsmath}
\usepackage{amssymb}
\usepackage{geometry}
\geometry{margin=1in}

\title{Guía de Métricas para Ciencia de Datos}
\author{Mario Ruiz}
\date{Noviembre 2025}

\begin{document}

\section*{Mean Absolute Error (MAE)}

\subsection*{Fórmula del MAE}
\[
MAE = \frac{1}{n} \sum_{i=1}^{n} |y_i - \hat{y}_i|
\]

\noindent
Donde:
\begin{itemize}
    \item $n$: número total de observaciones
    \item $y_i$: valor real
    \item $\hat{y}_i$: valor predicho por el modelo
\end{itemize}

\subsection*{Interpretación}
El \textbf{Mean Absolute Error (MAE)} mide el error promedio absoluto entre las predicciones y los valores reales.
Cada diferencia se toma en valor absoluto para evitar que los errores positivos y negativos se compensen entre sí.

\subsection*{Propiedades clave}
\begin{itemize}
    \item Siempre es \textbf{no negativa}: $MAE \ge 0$. Un valor de 0 indica predicciones perfectas.
    \item Usa la \textbf{distancia L1 (Manhattan)} como referencia geométrica.
    \item Es \textbf{robusta ante valores atípicos}, pues no eleva los errores al cuadrado.
    \item Mantiene las mismas unidades que la variable objetivo, facilitando su interpretación.
\end{itemize}

\subsection*{Resumen}
La MAE indica, en promedio, cuánto se equivoca un modelo por observación. 
Es ideal cuando se busca una medida de error intuitiva, estable y fácilmente comparable entre modelos.

\end{document}
