\documentclass[12pt]{article}
\usepackage[utf8]{inputenc}
\usepackage{amsmath}
\usepackage{amssymb}
\usepackage{geometry}
\geometry{margin=1in}

\title{Guía de Métricas para Ciencia de Datos}
\author{Mario Ruiz}
\date{Noviembre 2025}

\begin{document}

% \maketitle

\section*{Specificity (Especificidad)}

\subsection*{Fórmula principal}
\[
\text{Specificity} = \frac{TN}{TN + FP}
\]

\noindent
Donde:
\begin{itemize}
    \item $TN$: Verdaderos negativos — casos negativos correctamente clasificados.
    \item $FP$: Falsos positivos — casos negativos incorrectamente clasificados como positivos.
\end{itemize}

\subsection*{Componentes matemáticos}
\begin{itemize}
    \item \textbf{True Negatives (TN):} número de observaciones donde $y = 0$ y $\hat{y} = 0$.
    \item \textbf{False Positives (FP):} errores de “sobre-detección”, donde $y = 0$ pero $\hat{y} = 1$.
    \item \textbf{Denominador ($TN + FP$):} total de casos que realmente pertenecen a la clase negativa.
\end{itemize}

\subsection*{Interpretación matemática}
La especificidad representa una probabilidad condicional:
\[
P(\hat{y} = 0 \;|\; y = 0)
\]
Esto se interpreta como: “la probabilidad de que el modelo clasifique correctamente un caso negativo, dado que realmente es negativo”.

\subsection*{Rango e interpretación}
\begin{itemize}
    \item \textbf{Rango:} $[0, 1]$ (o $[0\%, 100\%]$)
    \item \textbf{Valor óptimo:} $1.0$ (modelo perfecto, sin falsos positivos)
    \item \textbf{Ejemplos:}
    \begin{itemize}
        \item $\text{Specificity} = 1.0$: nunca comete falsos positivos.
        \item $\text{Specificity} = 0.5$: acierta el 50\% de los casos negativos.
        \item $\text{Specificity} = 0.0$: clasifica todos los casos negativos como positivos.
    \end{itemize}
\end{itemize}

\subsection*{Relación con otras métricas}
La especificidad está estrechamente relacionada con la tasa de falsos positivos (FPR):
\[
\text{Specificity} = 1 - \text{FPR}
\quad \text{donde} \quad
\text{FPR} = \frac{FP}{FP + TN}
\]

\noindent
Esto significa que a medida que aumentan los falsos positivos, la especificidad disminuye en la misma proporción.

\subsection*{Resumen}
La \textit{Specificity} mide la capacidad del modelo para identificar correctamente los casos negativos.  
Es esencial en escenarios donde los falsos positivos son costosos o indeseables —por ejemplo, diagnósticos médicos, detección de fraudes o controles de calidad—.  
Junto con el \textbf{Recall}, proporciona una evaluación equilibrada del rendimiento de un modelo en ambas clases.

\end{document}
