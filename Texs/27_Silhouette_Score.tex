\documentclass[12pt]{article}
\usepackage[utf8]{inputenc}
\usepackage{amsmath}
\usepackage{amssymb}
\usepackage{geometry}
\geometry{margin=1in}

\title{Guía de Métricas para Ciencia de Datos}
\author{Mario Ruiz}
\date{Noviembre 2025}

\begin{document}

% \maketitle

\section*{Silhouette Score}

\subsection*{Definición general}
El \textbf{Silhouette Score} mide la calidad de un agrupamiento (\textit{clustering}) evaluando simultáneamente la \textit{cohesión interna} y la \textit{separación externa} de los clusters.  
Matemáticamente, cuantifica qué tan bien asignado está cada punto a su grupo.

\subsection*{Fórmula individual}
Para un punto de datos $i$, el coeficiente de silueta se define como:

\[
s(i) = \frac{b(i) - a(i)}{\max(a(i), b(i))}
\]

\noindent
Donde:
\begin{itemize}
    \item $a(i)$: Distancia promedio \textit{intra-cluster}.
    \[
    a(i) = \frac{1}{|C_I| - 1} \sum_{j \in C_I, j \neq i} d(i, j)
    \]
    Representa qué tan compacto es el cluster al que pertenece $i$.
    \item $b(i)$: Distancia promedio al cluster más cercano.
    \[
    b(i) = \min_{J \neq I} \left\{ \frac{1}{|C_J|} \sum_{k \in C_J} d(i, k) \right\}
    \]
    Mide la separación del punto $i$ respecto al cluster más próximo.
\end{itemize}

\noindent
Con:
\begin{itemize}
    \item $C_I$: Cluster al que pertenece el punto $i$.
    \item $|C_I|$: Número de puntos dentro del cluster $I$.
    \item $d(i,j)$: Distancia entre los puntos $i$ y $j$ (euclidiana, Manhattan, coseno, etc.).
\end{itemize}

\subsection*{Silhouette Score global}
El promedio general del coeficiente de silueta para todas las observaciones es:

\[
\text{Silhouette Score} = \frac{1}{n} \sum_{i=1}^{n} s(i)
\]

\noindent
Donde $n$ es el número total de puntos en el conjunto de datos.

\subsection*{Interpretación matemática}

\paragraph{Caso ideal ($s(i) \to 1$):}
\begin{itemize}
    \item $a(i) \to 0$ — el punto está muy cerca de otros puntos de su mismo cluster.
    \item $b(i)$ es grande — el punto está bien separado de otros clusters.
    \item Resultado: agrupamiento perfecto para este punto.
\end{itemize}

\paragraph{Caso problemático ($s(i) \to -1$):}
\begin{itemize}
    \item $a(i)$ es grande — el punto está lejos de los miembros de su propio cluster.
    \item $b(i) \to 0$ — el punto está más cerca de otro cluster.
    \item Resultado: el punto probablemente está mal asignado.
\end{itemize}

\paragraph{Caso límite ($s(i) \approx 0$):}
\begin{itemize}
    \item $a(i) \approx b(i)$ — el punto está equidistante entre su cluster y el vecino.
    \item Resultado: el punto se encuentra en la frontera entre clusters.
\end{itemize}

\subsection*{Propiedades matemáticas}
\begin{itemize}
    \item \textbf{Rango:} $s(i) \in [-1, 1]$
    \begin{itemize}
        \item $1$: Asignación óptima.
        \item $0$: Punto fronterizo.
        \item $-1$: Asignación incorrecta.
    \end{itemize}
    \item \textbf{Normalización:} La división por $\max(a(i), b(i))$ asegura escala uniforme e interpretación comparativa.
    \item \textbf{Balance:} Penaliza tanto baja cohesión interna como baja separación entre clusters.
    \item \textbf{Robustez:} Puede aplicarse con cualquier métrica de distancia.
\end{itemize}

\subsection*{Ventajas e interpretabilidad}
\begin{itemize}
    \item No requiere conocimiento previo del número “correcto” de clusters — puede compararse entre distintas configuraciones de $k$.
    \item Altamente interpretable: un valor cercano a $1$ indica un clustering bien definido, mientras que valores negativos sugieren mezclas o solapamientos entre grupos.
    \item Adecuado para evaluación global o por punto individual.
\end{itemize}

\subsection*{Conclusión}
El \textbf{Silhouette Score} combina cohesión y separación en una única métrica normalizada, ofreciendo una visión matemática y geométrica clara de la calidad del agrupamiento.  
Su simplicidad y robustez lo convierten en una de las métricas más utilizadas en evaluación no supervisada.

\end{document}
